% !TeX root = ../main.tex


\section{12 chapter}
Charismatic, transformational, and transactional leadership

\subsection{Attribution Theory of Charismatic Leadership} % (fold)
\label{sub:attribution_theory_of_charismatic_leadership}
By Conger and Kanungo (1987) - theory of charismatic leadership based on the assumption that charisma is an attributional phenomenon
\begin{itemize}
	\item Leader Behaviors
	\begin{itemize}
		\item Novel and Appealing Vision - Charisma is more likely to be attributed to leaders who advocate a vision that is highly discrepant from the status quo
		\item Emotional Appeals to Values - Followers are more likely to attribute charisma to leaders who inspire them with emotional appeals to their values and ideals
		\item Unconventional Behavior - Charisma is more likely to be attributed to leaders who act in unconventional ways to achieve the vision
		\item Self sacrifices - if they make self‐sacrifices for the benefit of followers, and they take personal risks or incur high costs to achieve the vision they espouse
		\item Confidence and Optimism - Leaders who appear confident about their proposals are more likely to be viewed as charismatic than leaders who appear doubtful and confused
	\end{itemize}
	\item Influence Processes - The primary influence process is personal identification 
		\\Charismatic leaders appear so extraordinary, due to their strategic insight, strong convictions, self‐confidence, unconventional behavior, and dynamic energy that subordinates idolize these leaders and want to become like them
	\item Facilitating Conditions - Contextual variables are especially important for charismatic leadership because the attribution of exceptional ability to a leader is rare and may be highly dependent upon characteristics of the situation. One important situational variable is follower fear and anxiety about the future
		\\Charismatic leaders are more likely to emerge in crisis situations where people are worried about economic loss, physical danger, or threats to core values
\end{itemize}

% subsection attribution_theory_of_charismatic_leadership (end)

\subsection{Self-Concept Theory of Charismatic Leadership} % (fold)
\label{sub:self_concept_theory_of_charismatic_leadership}
By House (1977) - theory to explain charismatic leadership in terms of a set of testable propositions

\begin{itemize}
	\item Leader Traits and Behaviors - Charismatic leaders are likely to have a strong need for power, high self‐confidence, and a strong conviction in their own beliefs and ideals
	\begin{itemize}
		\item articulating an appealing vision, 
		\item using strong, expressive forms of communication when articulating the vision, 
		\item taking personal risks and making self‐sacrifices to attain the vision,
		\item communicating high expectations, 
		\item expressing optimism and confidence in followers,
		\item modeling behaviors consistent with the vision, 
		\item managing follower impressions of the leader, 
		\item building identification with the group or organization,  
		\item empowering followers
	\end{itemize}
	\item Influence Processes
	\begin{itemize}
		\item Personal Identification - followers will imitate the leader’s behavior, carry out the leader’s requests, and make an extra effort to please the leader
		\item Social Identification - when people take pride in being part of the group or organization and regard membership as one of their most important social identities
		\item Internalization - followers embrace the leader’s mission or objectives as something that is worthy of their commitment
		\item Self and Collective Efficacy 
			\\self‐efficacy is the belief that one is competent and capable of attaining difficult task objectives
			\\Collective efficacy refers to the perception of group members that they can accomplish exceptional feats by working together
		\item Emotional Contagion - A leader who is very positive and enthusiastic can influence the mood of followers to be more positive, which is likely to increase their enthusiasm for the work and their perception that they can accomplish difficult objectives
	\end{itemize}
	\item Facilitating Conditions - charismatic leaders must be able to understand the needs and values of followers. In addition, it must be possible to 	define task roles in ideological terms that will appeal to followers
\end{itemize}
% subsection self_concept_theory_of_charismatic_leadership (end)

\subsection{Other Conceptions of Charisma} % (fold)
\label{sub:other_conceptions_of_charisma}
\begin{itemize}
	\item psychodynamic processes: intense personal identification. Regression + transference + projection
	\item social contagion: followers influence each other > spontaneous spread of emotional and behavioral reactions among a group of people
	\item close and distant charisma: amount of direct interaction between a leader and followers affects attribution of charisma. Distant charismatics > described in terms of substantive achievements and effects on follower political attitudes. Close charismatics > described in terms of their effects on follower motivation, task behavior, identification with leader

\end{itemize}
% subsection other_conceptions_of_charisma (end)


\subsection{Consequences of Charismatic Leadership} % (fold)
\label{sub:consequences_of_charismatic_leadership}

\subsubsection{Positive Charisma} % (fold)
\label{ssub:positive_charisma}
	Followers are more likely to experience psychological growth and development of their abilities, and the organization is more likely to adapt to an environment that is dynamic, hostile, and competitive.
	\\The positive charismatic can lead the organization in coping with a temporary crisis, but if prolonged for a long period of time, a single‐minded achievement culture creates excessive stress and causes psychological disorders for members who are unable to tolerate this stress.

% subsubsection positive_charisma (end)

\subsubsection{Negative Charisma} % (fold)
\label{ssub:negative_charisma}
	Charismatic leaders tend to make more risky decisions that can result in a serious failure, and they tend to make enemies who will use such a failure as an opportunity to remove the leader from office.

	\begin{itemize}
		\item Being in awe of the leader reduces good suggestions by followers.
		\item Desire for leader acceptance inhibits criticism by followers.
		\item Adoration by followers creates delusions of leader infallibility.
		\item Excessive confidence and optimism blind the leader to real dangers.
		\item Denial of problems and failures reduces organizational learning.
		\item Risky, grandiose projects are more likely to fail.
		\item Taking complete credit for successes alienates some key followers.
		\item Impulsive, nontraditional behavior creates enemies as well as believers.
		\item Dependence on the leader inhibits development of competent successors.
		\item Failure to develop successors creates an eventual leadership crisis.
	\end{itemize}
% subsubsection negative_charisma (end)
% subsection consequences_of_charismatic_leadership (end)


\subsection{Transformational Leadership} % (fold)
\label{sub:transformational_leadership}
	Transforming leadership appeals to the moral values of followers in an attempt to raise their consciousness about ethical issues and to mobilize their energy and resources to reform institutions. 
	\\Transactional leadership motivates followers by appealing to their self‐interest and exchanging benefits

	\begin{itemize}
		\item Transformational Behaviors
		\begin{itemize}
			\item Idealized influence - is behavior that increases follower identification with the leader, such as setting an example of courage and dedication and making self‐sacrifices to benefit followers
			\item Individualized consideration - includes providing support, encouragement, and coaching to followers
			\item Inspirational motivation - communicating an appealing vision, and using symbols to focus subordinate effort
			\item Intellectual stimulation - is behavior that influences followers to view problems from a new perspective and look for more creative solutions
		\end{itemize}
		\item Transactional Behaviors
		\begin{itemize}
			\item Contingent reward - clarification of accomplishments necessary to obtain rewards, and the use of incentives to influence subordinate task motivation
			\item Active management by exception - looking for mistakes and enforcing rules to avoid mistakes
			\item Passive management by exception - use of contingent punishments and other corrective action 
		\end{itemize}
	\end{itemize}

% subsection transformational_leadership (end)

\subsection{Comparison of Charismatic and Transformational Leadership} % (fold)
\label{sub:comparison_of_charismatic_and_transformational_leadership}
	Many of the leadership behaviors in the theories of charismatic and transformational leadership appear to be the same, but some important differences are evident as well. Transformational leaders probably do more things that will empower followers and make them less dependent on the leader, such as delegating significant authority to individuals or teams, developing follower skills and self‐confidence, providing direct access to sensitive information, eliminating unneces- sary controls, and building a strong culture to support empowerment. Charismatic leaders probably do more things that foster an image of extraordinary competence for the leader and increase subordinate dependence, such as impression management, information restriction, unconventional behavior, and personal risk taking.
	\\difference between transformational and charismatic leadership involves how often each type of leadership occurs and the facilitating conditions for it. According to Bass, transformational leaders can be found in any organization at any level, and this type of leadership is universally relevant for all types of situations (Bass, 1996, 1997). In contrast, charismatic leaders are rare, and their emergence appears to be more dependent on unusual conditions. The reactions to charismatics are usually more extreme and diverse than reactions to transformational leaders (Bass,1985). The affective reaction aroused by charismatics often polarizes people into opposing camps of loyal supporters and hostile opponents
% subsection comparison_of_charismatic_and_transformational_leadership (end)

\subsection{Guidelines for Inspirational Leadership} % (fold)
\label{sub:guidelines_for_inspirational_leadership}

\begin{itemize}
	\item Articulate a clear and appealing vision - Transformational leaders strengthen the existing vision or build commitment to a new vision. A clear vision of what the organization could accomplish or become helps people understand the purpose, objectives, and priorities of the organization
	\item Explain how the vision can be attained - leader must also convince followers that the vision is feasible. It is important to make a clear link between the vision and a credible strategy for attaining it.
	\item Act confident and optimistic - It is important to remain optimistic about the likely success of the group in attaining its vision, especially in the face of temporary roadblocks and setbacks. A leader’s confidence and optimism can be highly contagious.
	\item Express confidence in followers - The motivating effect of a vision also depends on the extent to which subordinates are confident about their ability to achieve it. People perform better when a leader has high expectations for them and shows confidence in them
	\item Use dramatic, symbolic actions to emphasize key values - Concern for a value or objective is demonstrated by the way a manager spends time, by resource allocation decisions made when trade‐offs are necessary between objectives, by the questions the manager asks, and by what actions the manager rewards
	\item Lead by example - One way a leader can influence subordinate commitment is by setting an example of exemplary behavior in day‐to‐day interactions with subordinates. It is especially important for actions that are unpleasant, dangerous, unconventional, or controversial
\end{itemize}
% subsection guidelines_for_inspirational_leadership (end)

\subsection{Key Terms} % (fold)
\label{sub:key_terms}
charisma
charismatic leadership
emotional contagion
personal identification
role modeling
self‐concept
self‐efficacy
self‐identity
social identification
symbolic action
transactional leadership
transformational leadership
vision

% subsection key_terms (end)