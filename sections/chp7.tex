% !TeX root = ../main.tex


\section{7 chapter}
Contingency Theories andAdaptive Leadership


\subsection{General Description of Contingency Theories} % (fold)
\label{ssub:general_description_of_contingency_theories}
Contingency theories describe how aspects of the leadership situation can alter a leader’s influence and effectiveness

\subsubsection{Types of Variables} % (fold)
\label{ssub:types_of_variables}

\begin{itemize}
	\item Situation directly effects Outcomes or Mediators - When a situational variable can make a mediating variable or an outcome more  favorable, it is sometimes called a “substitute” for leadership. An example is when subordinates have extensive prior training and experience.
	\item Situation directly influences Leader Behavior - Aspects of the situation such as formal rules, policies, role expectations, and organizational values can encourage or constrain a leader’s behavior, and they are sometimes called demands and constraints
	\item Situation moderates Effects of Leader Behavior - For example providing coaching will have a stronger impact on subordinate performance when the leader has relevant expertise. This expertise enables the leader to provide better coaching, and subordinates are more likely to follow advice from a leader who is perceived to be an expert 

\end{itemize}

% subsubsection types_of_variables (end)

% subsection general_description_of_contingency_theories (end)

\subsection{Early Contingency Theories} % (fold)
\label{ssub:early_contingency_theories}

\subsubsection{Path-Goal Theory} % (fold)
\label{ssub:path_goal_theory}
Describes how a leader’s task-oriented behavior (“instrumental leadership”) and relations-oriented behavior (“supportive leadership”) influence
subordinate satisfaction and performance in different situations (Evans, 1970; House, 1971), Participative leadership and achievement-oriented leadership (e.g., Evans, 1974; House, 1996; House \& Mitchell, 1974).
\\According to path-goal theory, the effect of leader behavior on subordinate satisfaction and effort depends on aspects of the situation, including task characteristics and subordinate characteristics

\begin{itemize}
	\item behaviors
	\begin{itemize}
		\item \textbf{supportive leadership}: giving consideration to needs and concerns of subordinates and their welfare, establish friendly climate
		\item \textbf{directive leadership}: giving specific guidance, asking to follow rules and procedures, scheduling and coordinating the work
		\item \textbf{participative leadership}: consulting with subordinates and taking their opinions and suggestions into account
		\item \textbf{achievement­oriented leadership}: setting challenging goals, seeking better performance, emphasize excellence, showing confidence to subordinates for high performance
	\end{itemize}
	\item variables
	\begin{itemize}
		\item when the task is boring, stressful, dangerous, supportive leadership leads to increased subordinate effort and satisfaction
		\item when the task is unstructured, complex, the subordinates are inexperienced, and there is little formalization of rules and procedures, then directive leadership will result in higher subordinate satisfaction and effort.
	\end{itemize}
	\item limitations:
	\begin{itemize}
		\item relies on expectancy theory that does not take in consideration emotional reactions to decision dilemmas and some important aspects of human motivation such as self­concepts.
		\item relationships are unreal
		\item each type of leadership behavior is considered separately
	\end{itemize}
\end{itemize}
% subsubsection path_goal_theory (end)

\subsubsection{leadership substitutes theory} % (fold)
\label{ssub:leadership_substitutes_theory}
a model to identify aspects of the situations that reduce the importance of leadership by managers and other
formal leaders.
\begin{itemize}
	\item substitutes: make leader behavior unnecessary and redundant vs. neutralizers: are characteristics of task or organization that prevent a leader from acting in a specified way or that nullify the effects of the leader’s actions
	\item 2 variables: role clarity and task motivation
	\item 2 solutions:
	\begin{itemize}
		\item to remove neutralizers to make situation more favorable for the leader
		\item to make leadership less important by increasing substitutes
	\end{itemize}
	\item limitations
	\begin{itemize}
		\item lack of a description of explanatory processes that would help differentiate between substitutes that reduce the importance of an intervening variable and substitutes that involve leadership behavior by people other than the leader
		\item use of too broadly defined behavior categories such as supportive and instrumental leadership

	\end{itemize}
\end{itemize}

% subsubsection leadership_substitutes_theory (end)

\subsubsection{situational leadership theory} % (fold)
\label{ssub:situational_leadership_theory}
	It specifies the appropriate type of leadership behavior for a subordinate in various situations. Behavior was defined in terms of directive and supportive leadership and decision procedures

\begin{itemize}
	\item appropriate leadership behavior for different maturity levels of subordinates
	\begin{itemize}
		\item high maturity: has both the ability and self­confidence to do a task
		\item low maturity: lacks ability and self­confidence
	\end{itemize}

\item for low maturity subordinate - leader: task­oriented behavior (be directive in clarifying roles, standards and procedures)
\item moderate maturity - leader: decrease task­oriented and increase relations­oriented behavior (act supportive, consult
with the subordinate, provide praise and attention)
\item high maturity - leader: use low level of both behavior. Subordinate has the ability and self­confidence to do tasks
\end{itemize}
% subsubsection situational_leadership_theory (end)

\subsubsection{the LPC contingency model} % (fold)
\label{ssub:the_lpc_contingency_model}
	LPC Contingency Model describes how the situation moderates the effects on group performance of a leader trait called the least preferred coworker (LPC) score.

	Fiedler (1978) : LPC scores reveal a leader’s motive hierarchy.
	\begin{itemize}
		\item A high LPC leader is strongly motivated to have close, interpersonal relationships and will act in a considerate, supportive manner if relationships need to be improved
		\item a Low LPC leader - is primarily motivated by achievement of task objectives and will emphasize task-oriented behavior whenever task problems arise
	\end{itemize}
	Rice (1978) : low LPC score value task achievement more than interpersonal relations, whereas leaders with high LPC scores value interpersonal relations more than task achievement.
	\begin{itemize}
		\item high LPC scores value interpersonal relations more than task achievement
		\item low LPC score value task achievement more than interpersonal relations
	\end{itemize}
	Situational favorability - The relationship between a leader’s LPC score and group performance depends on a complex situational variable called situational favorability.
	\\The situation is most favorable when the leader has substantial position power, the task is highly structured, and relations with subordinates are good

% subsubsection the_lpc_contingency_model (end)

\subsubsection{cognitive resources theory} % (fold)
\label{ssub:cognitive_resources_theory}
	Describes the conditions under which cognitive resources such as intelligence and experience are related to group performance.


	performance of a leader’s group is determined by interaction between
	\begin{itemize}
		\item two leader traits (intelligence, experience)
		\item one leader behavior (directive leadership)
		\item two aspects of the leadership situation (interpersonal stress and the nature of the group’s task)
	\end{itemize}

	Interpersonal stress for the leader moderates the relation between leader intelligence and subordinate performance. Stress may be due to a boss who creates role conflict or demands miracles without providing necessary support and resources. Other sources of stress include frequent work crises and serious conflicts with subordinates.
\begin{itemize}
	\item high stress on leaders:
	\begin{itemize}
		\item no relationship (or negative) between leader intelligence and decision quality
		\item experienced leaders rely on experience under high stress
	\end{itemize}
	\item low stress on leaders:
	\begin{itemize}
		\item high intelligence results in good plans and decisions
		\item experienced leaders rely on intelligence under low stress
	\end{itemize}

\end{itemize}

\begin{itemize}
	\item leaders with little experience rely on intelligence under both situations
	\item limitations:
	\begin{itemize}
		\item no explicit rationale is given for use of general intelligence rather than specific cognitive skills.
		\item only one leadership behavior
	\end{itemize}
\end{itemize}

% subsubsection cognitive_resources_theory (end)

\subsubsection{multiple-linkage model} % (fold)
\label{ssub:multiple_linkage_model}
	Describes how managerial behavior and situational variables jointly influence the performance of individual subordinates and the leader’s work unit. 

	Four types of variables:
	\begin{itemize}
		\item Mediating Variables
		\begin{itemize}
			\item task commitment: influenced by formal reward system and properties of the work
			\item ability of members: influence by recruitment and selection system and prior training/experience of members
			\item role clarity: affected by task structure, prior member experience, and external dependencies
			\item work group organisation: type of technology used, competitive strategy of the organization
			\item cooperation and teamwork: affected by size of group, stability of membership, similarity among members, reward system, and organization of work
			\item adequacy or resources: affected by the organization’s formal budgetary systems, procurement systems, and inventory control systems, as well as economic conditions at time
			\item external coordination: influenced by the formal structure of the organization
		\end{itemize}
		\item Situational Variables - directly influence mediating variables and can make them either more or less favorable
		\item Managerial Behaviors
		\item Criterion Variables - leadership emergence, advancement, or effectiveness; subjective or objective measures
	\end{itemize}
\begin{itemize}
	\item short­term actions to correct deficiencies (= neutralizers)
	\begin{itemize}
		\item leaders may influence group members to work faster or do better work
			\item  may increase member ability to do the work
			\item  may organize and coordinate activities in a more efficient way
			\item  may obtain resources needed immediately to do the work
			\item  may act to improve external coordination by meeting with outsiders or to plan activities and resolve conflicting demands on the work unit
	\end{itemize}
	\item long­term effects for improvements in group performance. Effective leaders do :
	\begin{itemize}
		\item long-term improvement programs to upgrade equipment, and facilities
		\item to increase the level of employee skills and commitment.
		\item reducing dependence on unreliable sources.
		\item Initiate new, more profitable activities for the work unit
	\end{itemize}			
\end{itemize}

% subsubsection multiple_linkage_model (end)

\subsubsection{Conceptual Weaknesses in Contingency Theories} % (fold)
\label{ssub:conceptual_weaknesses_in_contingency_theories}

\begin{itemize}
	\item Over-emphasis on Behavior Meta-categories
	\item Ambiguous Description of Relationships - Most of the contingency theories do not clearly indicate whether the form of the relationship between the independent variable and the dependent variable changes as the situational variable increases
	\item Inadequate Explanation of Causal Effects - Most contingency theories do not provide an adequate explanation of the underlying reasons for the proposed relationships
	\item Lack of Attention to Behavior Patterns - Most contingency theories explain only the separate, independent effects of each type of leadership behavior included in the theory.
	\item Lack of Attention to Joint Effects of Situational Variables - Most contingency theories do not explicitly consider how multiple situational variables interact in their moderating effects
	\item Failure to Distinguish Between Mediators and Situational Moderators
\end{itemize}

% subsubsection conceptual_weaknesses_in_contingency_theories (end)

\subsubsection{Guidelines for Adaptive Leadership} % (fold)
\label{ssub:guidelines_for_adaptive_leadership}

\begin{itemize}
	\item Understand your leadership situation and try to make it more favorable.
	\item Increase flexibility by learning how to use a wide range of relevant behaviors.
	\item Use more planning for a long, complex task.
	\item Consult more with people who have relevant knowledge.
	\item Provide more direction to people with interdependent roles.
	\item Monitor a critical task or unreliable person more closely.
	\item Provide more coaching to an inexperienced subordinate.
	\item Be more supportive to someone with a highly stressful task.
\end{itemize}

% subsubsection guidelines_for_adaptive_leadership (end)

\subsubsection{Guidelines for Managing Immediate Crises} % (fold)
\label{ssub:guidelines_for_managing_immediate_crises}
\begin{itemize}
	\item Anticipate problems and prepare for them.
	\item Learn to recognize early warning signs for an impending problem.
	\item Quickly identify the nature and scope of the problem.
	\item Direct the response by the unit or team in a confident and decisive way.
	\item Keep people informed about a major problem and what is being done to resolve it.
	\item Use a crisis as an opportunity to make necessary changes.
\end{itemize}
% subsubsection guidelines_for_managing_immediate_crises (end)

\subsubsection{Key Terms} % (fold)
\label{ssub:key_terms}
cognitive resources theory
contingency theories
directive leadership
mediating variable
moderator variable
multiple-linkage model
normative decision model
neutralizer
path-goal theory
situational leadership
theory
substitute for leadership
% subsubsection key_terms (end)
% subsection early_contingency_theories (end)
