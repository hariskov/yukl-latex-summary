% !TeX root = ../main.tex


\section{Third chapter}
Trait and behavioral theories of leadership

\subsection{Ways for Describing Leadership Behavior} % (fold)
\label{sub:ways_for_describing_leadership_behavior}

\subsection{Major Types of Leadership Behavior} % (fold)
\label{ssub:major_types_of_leadership_behavior}

\begin{itemize}
	\item Task Behaviors - is primarily concerned with accomplishing the task in an efficient and reliable way
	\begin{itemize}
		\item Organize work activities to improve efficiency.
		\item Plan short-term operations.
		\item Assign work to groups or individuals.
		\item Clarify what results are expected for a task.
		\item Explain priorities for different task objectives.
		\item Set specific goals and standards for task performance.
		\item Explain rules, policies, and standard operating procedures.
		\item Direct and coordinate work activities.
		\item Monitor operations and performance.
		\item Resolve immediate problems that would disrupt the work.
	\end{itemize}
	\item Relations Behaviors - is primarily concerned with increasing mutual trust, cooperation, job satisfaction, and identification with the team or organization
	\begin{itemize}
		\item Provide support and encouragement to someone with a difficult task.
		\item Express confidence that a person or group can perform a difficult task.
		\item Socialize with people to build relationships.
		\item Recognize contributions and accomplishments.
		\item Provide coaching and mentoring when appropriate.
		\item Consult with people on decisions affecting them.
		\item Empower people to determine the best way to do a task.
		\item Keep people informed about actions affecting them.
		\item Help resolve conflicts in a constructive way.
		\item Use symbols, ceremonies, rituals, and stories to build team identity.
		\item Encourage mutual trust and cooperation among members of the work unit.
		\item Recruit competent new members for the team or organization.
	\end{itemize}
	\item Change-oriented Behavior - behavior is primarily concerned with understanding the environment, finding innovative ways to adapt to it, and implementing major changes in strategies, products, or processes
	\begin{itemize}
		\item Monitor the external environment to detect threats and opportunities.
		\item Interpret events to explain the need for change.
		\item Study competitors and outsiders to get ideas for improvements.
		\item Envision exciting new possibilities for the organization.
		\item Encourage people to view problems or opportunities in a different way.
		\item Develop innovative new strategies linked to core competencies.
		\item Encourage and facilitate innovation and entrepreneurship in the organization.
		\item Encourage and facilitate collective learning in the team or organization.
		\item Experiment with new approaches for achieving objectives.
		\item Make symbolic changes that are consistent with a new vision or strategy.
		\item Encourage and facilitate efforts to implement major change.
		\item Announce and celebrate progress in implementing change.
	\end{itemize}
	\item Participative Leadership - It involves a leader’s use of decision procedures that allow other people such as subordinates to have some influence over decisions that will affect them
	\item Transformational Leadership
		\begin{itemize}
			\item relations-oriented behaviors such as supporting and developing
			\item change-oriented behaviors such as articulating an appealing vision and encouraging innovative thinking
		\end{itemize}
	\item External Leadership Behaviors
	\begin{itemize}
		\item Networking - building and maintaining favorable relationships with peers, superiors, and outsiders
		\item Environmental scanning - collecting information about relevant events and changes in the external environment, identifying threats and opportunities for the leader’s group or organization, and identifying best practices that can be imitated or adapted
		\item Representing - lobbying for resources and assistance from superiors, promoting and defending the reputation of the leader’s group or organization, negotiating agreements with peers and outsiders such as clients and suppliers, and using political tactics to influence decisions made by superiors or governmental agencies
	\end{itemize}
\end{itemize}

% subsection major_types_of_leadership_behavior (end)

\subsection{Methods for Studying the Effects of Leader Behavior} % (fold)
\label{ssub:methCritical Incident Study}

	\begin{itemize}
		\item Critical Incident Study
		\item Diary Incident Study
		\item Field Experiments
	\end{itemize}

% subsection methods_for_studying_the_effects_of_leader_behavior (end)

\subsection{Planning Work Activities} % (fold)
\label{ssub:planning_work_activities}
Short-term planning of work activities means deciding what to do, how to do it, who will do it, and when it will be done. The purpose of planning is to ensure efficient organization of the work unit, coordination of activities, and effective utilization of resources.

\begin{itemize}
	\item Identify necessary action steps.
	\item Identify the optimal sequence of action steps.
	\item Estimate the time needed to carry out each action step.
	\item Determine starting times and deadlines for each action step.
	\item Estimate the cost of each action step.
	\item Determine who will be accountable for each action step.
	\item Develop procedures for monitoring progress.
\end{itemize}
% subsection planning_work_activities (end)

\subsection{Clarifying Roles and Objectives} % (fold)
\label{ssub:clarifying_roles_and_objectives}
Clarifying is the communication of plans, policies, and role expectations. Major subcategories of clarifying include (1) defining job responsibilities and requirements, (2) setting performance goals, and (3) assigning specific tasks

\begin{itemize}
	\item Clearly explain an assignment.
	\item Explain the reason for the assignment.
	\item Check for understanding of the assignment.
	\item Provide any necessary instruction in how to do the task.
	\item Explain priorities for different objectives or responsibilities.
	\item Set specific goals and deadlines for important tasks.
\end{itemize}

% subsection clarifying_roles_and_objectives (end)


\subsection{Monitoring Operations and Performance} % (fold)
\label{ssub:monitoring_operations_and_performance}

Monitoring involves gathering information about the operations of the manager’s organizational unit, including the progress of the work, the performance of individual subordinates,the quality of products or services, and the success of projects or programs

\begin{itemize}
	\item Identify and measure key performance indicators.
	\item Monitor key process variables as well as outcomes.
	\item Measure progress against plans and budgets.
	\item Develop independent sources of information.
	\item Conduct progress review meetings at appropriate times.
	\item Observe operations directly when it is feasible.
	\item Ask specific questions about the work.
	\item Encourage reporting of problems and mistakes.
	\item Use information from monitoring to guide other behaviors.
\end{itemize}

% subsection monitoring_operations_and_performance (end)

\subsection{Supportive Leadership} % (fold)
\label{ssub:supportive_leadership}
Includes a wide variety of behaviors that show consideration, acceptance, and concern for the needs and feelings of other people. Supportive leadership helps to build and maintain effective interpersonal relationships


\begin{itemize}
	\item Show acceptance and positive regard - Supportive leadership means being polite and considerate. Maintain a pleasant, cheerful disposition
	\item Provide sympathy and support when the person is anxious or upset - Show understanding and sympathy for someone who is upset by stress and difficulties in the work. Take time to listen to the person’s concerns
	\item Bolster the person’s self-esteem and confidence - Indicate that the person is a valued member of the organization
	\item Be willing to help with personal problems - when assistance is requested or it is clearly needed because the person’s performance is being adversely affected
\end{itemize}
% subsection supportive_leadership (end)

\subsection{Developing Subordinate skills} % (fold)
\label{ssub:developing_subordinate_skills}
Developing includes several managerial practices that are used to increase a subordinate’s skills and facilitate job adjustment and career  advancement. Key component behaviors include mentoring, coaching, and providing developmental opportunities

\begin{itemize}
	\item Show concern for each person's development.
	\item Help the person identify ways to improve performance.
	\item Be patient and helpful when providing coaching.
	\item Provide helpful career advice.
	\item Help the person prepare for a job change.
	\item Encourage attendance at relevant training activities.
	\item Provide opportunities to learn from experience.
	\item Encourage coaching by peers when appropriate.
	\item Promote the person’s reputation.
\end{itemize}
% subsection developing_subordinate_skills (end)


\subsection{Providing Praise and Recognition} % (fold)
\label{ssub:providing_praise_and_recognition}
Recognizing involves giving praise and showing appreciation to others for effective performance, significant achievements, and important contributions to the organization.
Three major forms of recognizing are praise, awards, and recognition ceremonies

\begin{itemize}
	\item Recognize a variety of contributions and achievements.
	\item Actively search for contributions to recognize.
	\item Recognize improvements in performance.
	\item Recognize commendable efforts that failed.
	\item Do not limit recognition to high-visibility jobs.
	\item Do not limit recognition to a few best performers.
	\item Provide specific recognition.
	\item Provide timely recognition.
	\item Use an appropriate form of recognition.
\end{itemize}
% subsection providing_praise_and_recognition (end)


\subsection{Key Terms} % (fold)
\label{ssub:key_terms}
\begin{itemize}
	\item change-oriented behavior
	\item clarifying
	\item consideration
	\item critical incidents
	\item developing
	\item high-high leader
	\item initiating structure
	\item meta-categories
	\item monitoring
	\item networking
	\item participative leadership
	\item planning
	\item recognizing
	\item relations-oriented behavior
	\item supportive leadership
	\item task-oriented behavior
	\item transformational leadership
\end{itemize}


% subsection key_terms (end)
